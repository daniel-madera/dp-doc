% !TeX spellcheck = cs_CZ
\documentclass[a4paper,11pt,titlepage,fleqn]{article}

\usepackage[utf8]{inputenc}
\usepackage[top=3cm, bottom=2.5cm, left=3.5cm, right=2.5cm]{geometry}
\usepackage[czech]{babel}
\usepackage[IL2]{fontenc}  
\usepackage{graphicx}
\usepackage[nonumberlist,acronym]{glossaries}
\usepackage{cite}
\usepackage{fancyhdr}
\usepackage{afterpage}
\usepackage[hidelinks,unicode,hyperfootnotes]{hyperref}
\usepackage{footnote}
\usepackage{parskip}
\usepackage{setspace} 
\usepackage{listings}
\usepackage{pdfpages}
\usepackage{dirtree}
\usepackage{booktabs}
%\usepackage[T1]{fontenc}
%\usepackage[bottom]{footmisc}
%\usepackage{fancyvrb}

\lstset{
    language=PHP,
    basicstyle=\ttfamily\small,
    breaklines=true,
    prebreak=\raisebox{0ex}[0ex][0ex]{\ensuremath{\hookleftarrow}},
    frame=lines,
    showtabs=false,
    showspaces=false,
    showstringspaces=false,
    keywordstyle=\color{red}\bfseries,
    stringstyle=\color{green!50!black},
    commentstyle=\color{gray}\itshape,
    numbers=left,
    captionpos=t,
    escapeinside={\%*}{*)}
}

\renewcommand{\lstlistingname}{Ukázka kódu}
\renewcommand*{\lstlistlistingname}{Seznam zdrojových kódů}

\makeglossaries
\renewcommand*{\glsgroupskip}{}

%\fancyhead[RE,RO]{\textsc{\nouppercase{\leftmark}}}
\rhead{\textsc{\nouppercase{\leftmark}}}
\lhead{}
\pagestyle{fancy}

\addto\captionsczech{\def\refname{Použitá literatura}}
\linespread{1.3}
\setlength{\headheight}{15pt}

\newglossaryentry{tulg}{
    name={Technická univerzita v~Liberci},
    description={Technická univerzita v~Liberci}
}
\newglossaryentry{dp}{
    type=\acronymtype,
    name={DP},
    description={Diplomová práce},
    first={DP}
}

\begin{document}
\includepdf[pages={1,2,3}]{dp-titlepage.pdf}
\includepdf{prohlaseni.pdf}
\setcounter{page}{3}

\newpage
\thispagestyle{plain}
\section*{Abstrakt}
% 250 až 500 slov

\section*{Klíčová slova}
% 5 klíčových slov

\thispagestyle{empty}
\newpage

\section*{Abstract}
% 250 až 500 slov

\section*{Keywords}
% 5 klíčových slov

\thispagestyle{empty}

\newpage
\setcounter{tocdepth}{2}
\tableofcontents

\newpage
\listoffigures
\listoftables
\lstlistoflistings

\newpage
\printglossary[type=\acronymtype,title=Seznam zkratek]
%\printglossary[style=altlist,title=Slovník]
\cleardoublepage


\section{Úvod}
    % proč se zaobývat tímto tématem
    % neopakovat abstrakt, lehce nastínit zadání, jaká je motivace
    % popsat, jak je práce strukturovaná - rozcestník


\newpage
\section{Analýza}
    Výuka cizích jazyků je pro aktuální společnost jedna z nejzásadnějších otázek, ať se jedná o pracovní příležitosti v zahraničí, tak sociální problematika světa. Žáci a studenti se často účastní různých kroužků nebo později studenti využívají Erasmus programů, kde vyhledávají právě zlepšení komunikace v cizím jazyce. 

    \subsection{Hlavní cíle}
        % jak si aplikaci představuji
        % zlepšení přípravy na hodiny cizího jazyka
        % shrnout v odstavci požadavky na aplikaci
        
        Hlavním cílem aplikace je připravit žáky na hodinu cizího jazyka a zároveň přirozeně rozvíjet slovní zásobu, tak aby studenti neztratili motivaci a chuť k poznávání nových výrazů. Dále také umožnit žákům se protestovat a ověřit, zda naučenou sadu slov ovládají. Aplikace by se měla adaptovat na zdatnost a úroveň každého ze žáků. Vyučujícím by aplikace měla usnadnit správu a import slovíček, které třída má umět v rámci dané učebnice a následně předkládat žákům vhodnou slovní zásobu například pro následující lekci.

        \subsubsection{Personalizace}
            % personalizace podle potřeby konkrétní třídy - konrétní učebnice
            % personalizace podle potřeby jednotlivých žáků - generování na základě předcházejících výsledků
            Důležitým cílem aplikace by měla být personalizace podle potřeby jednotlivých žáků a celých tříd. Personalizace na úrovni žáka znamená přistupovat individuálně na základě jeho úrovně znalostí a dovedností. V případě konkrétní třídy jde o individuální přístup a to zejména v sadě slov, které je třeba v přípravě procvičovat. 

        \subsubsection{Motivace}
            Důležitou součástí vzdělávání obecně je motivace. Tedy přimět žáky, aby z vlastní iniciativy chtěli rozvíjet své vědomosti. Problém ale je, že se děti přirozeně neučí z vlastní iniciativy, ale aby uspokojili okolí. Motivace se rozdělují do skupin - vnitřní a vnější nebo pozitivní a negativní \cite{bib:motivace}. V analýze bude zaměřeno pouze na motivační prostředky, které lze zařadit do aplikace pro procvičování slovíček.

            Motivace založená na základě vlastních úspěchů je důležitá pro utvrzení sebevědomí žáků. Ocenění v případě zvládnutí sady slov nebo gramatického bloku lze v případě aplikace implementovat například hláškami s projevem pochvaly nebo jiným upozorněním na dosažený výsledek. Zajímavým prvkem v aplikaci by mohlo být i herní prostředí. Žáci si rádi hrají a již J. A. Komenský poukázal na důležitost her ve vzdělávacím procesu.

            % motivace dětí k učení slov (vrámci třídy)
            % motivace - přispění k úspěchu třídy
            % motivace - vlastní úspěchy

            Jedním z dalších stěžejních faktorů motivace je kolektiv. Právě díky kolektivu, ve kterým funguje přirozená rivalita, jsou schopni žáci dosáhnout mnohem vyšších výsledků než kdyby se vzdělávali odděleně a samostatně. Rivalita a soutěživost může projevovat i velmi negativním způsobem. Místo kamarádských vztahů mezi dětmi mohou vznikat nepřátelské, kde může docházet i k posměchu těch, kteří nemusí mít pro výuku až takové nadání. Zajímavější motivací tedy pro kolektiv je například vidina společně dosažených výsledků. V případě učení slovíček by to byl počet slovíček naučených za rok jako celá třída. Dochází zde k utužování kolektivu a děti by mohlo těšit to, že nějakým způsobem přispívají k úspěchům celé třídy.

        \subsubsection{Využití IT pro výuku}
            Posledních několik let se společnost ubírá trendem informačních technologií. Každý ze žáků má už od útlého věku přístup k počítači nebo k chytrému telefonu. Orientace a schopnost tyto zařízení používat není pro ně žádný problém. Přirozeně se tedy tyto zařízení postupně stávají součástí každodenní přípravy žáka na následující školní den. V některých případech tyto zařízení plně nahrazují klasické učebnice a jsou přímo začleněny do výuky. Použití informačních technologií má za následek i zlepšení motivace při výuce. Obecně je známo, že žáci raději studují slovíčka interaktivní formou hádanek, křížovek nebo her, než nekonečných seznamů slov.

            Důležitost cizích jazyků se projevuje na míře používání například chytrých tabulí nebo tabletů při výuce. Tyto zařízení umožňují interaktivní výuku, kde lze využít nejen textových, ale také obrázkových a zvukových prostředků pro lepší zasazení nově nabytých vědomostí do kontextu. 

            % výuka cizích jazyků - smartboards            
            % vysoká motivace dětí pracovat s PC

    \subsection{Existující řešení}
        Na trhu lze nalézt nepřeberné množství aplikací pro výuku cizích jazyků. Řada z nich jsou téměř komplexní systémy, které provází studenta od základních frází a slovíček až po gramatické standardy cizího jazyka. Analýza existujících řešení byla zaměřena na aplikace, které se zabývají především testováním slovíček a frází.

        Do analýzy existujících řešení byly zahrnuty tři desktopové, jedna webová a jedna mobilní aplikace.

        \subsubsection{TS Angličtina}
            Z analyzovaných řešení se jevila desktopová aplikace TS Angličtina od firmy Terasoft ta nejlépe funkčně propracovaná. Tato firma se zabývám širokou škálou výukových nástrojů se zaměřením na základní školy. V případě cizích jazyků se zabývají výukou anglického a německého jazyka. Hlavní předností aplikace je podpora nejvíce používaných učebnic cizího jazyka. Aplikace umožňuje testování různými způsoby - psaný překlad slova, porozumění mluvenému slovu, vybírání správných významů nebo doplňování vynechaných slov \cite{bib:terasoft}. Analýza byla tvořena pouze z informací vydavatele. Bohužel firma Terasoft neposkytuje DEMO nebo TRIAL verzi aplikace, která by hlubší analýzu umožňovala.

        \subsubsection{Langsoft Teacher}
            Dalším analyzovaným řešením byla aplikace Langsoft Teacher, která je dostupná v podání desktopové aplikace, ale také i mobilní aplikace pro platformy iOS a Android. Aplikace je velmi komplexní, obsahuje různé moduly pro testování například v obrazové formě pro předškolní děti. Zajímavá vlastnost, kterou aplikace disponuje, je pamatování problematických slov a nabízení těchto slov častěji než těch bezproblémových. Dále program umožňuje automaticky rozšiřovat slovní zásobu, která je v testování zahrnuta \cite{bib:langsoft}.

        \subsubsection{Duolingo}
            Dalším v pořadí byla aplikace Duolingo. Jedná se o mobilní aplikaci pro Android. Zahrnuje učivo cizího jazyka od základních komunikačních frází až po tvorbu gramaticky složitějších vět. V aplikaci je připravena dlouhá řada cizích jazyků - němčina, angličtina, španělština (kastilština), italština a další. Chybí ale více referenčních jazyků. Aktuálně lze využít pouze angličtinu. Aplikace kromě standardních funkcí zahrnuje i rozpoznávání mluvených odpovědí. Zajímavým poznatkem byl systém motivace uživatelů, kde si každý mohl pozvat své přátele, mezi kterými docházelo k sdílení dosažených výsledků. Dalším motivačním základem bylo nutkání udržení plánu pravidelného testování, jelikož v opačném případě docházelo k automatickému zvyšování objemu testovacích dat. Nevýhodou aplikace byla nutnost připojení k internetu. V případě požití mobilních dat, docházelo ke zpoždění zejména při rozpoznání slov.

        \subsubsection{Vocabulary Trainer}
            Vocabulary Trainer je webová aplikace zdarma napsaná v jazyce PHP, která naučí 5000 nejvíce používaných slov daného jazyka. Aplikace umožňuje hodně možného nastavení. K dispozici je i řada jazyků včetně češtiny a to jako referenční i jako učený jazyk. Testování spočívá nejdříve ve čtení slov a následně uživatel vybírá možnosti odpovědi. Aktuálně překládané slovo si lze kdykoliv přehrát v různé rychlosti. Jako motivační základ aplikace využívá jednoduchý bodový systém. Součástí je i kalendář s email upomínkou k dalšímu testování. Program je propracovaný, ale poměrně pomalý a dlouho trvá zejména úvodní načítání dat. Její výrobce LanguageCourse S.L. poskytuje i mobilní aplikaci pro Android k učení anglických slov a frází.

        \subsubsection{Supermemo aplikace}
            Za zmínku ještě stojí aplikace Supermemo. Není to aplikace s připravenými daty k testování slov cizího jazyka, ale slouží čistě jako šablona pro testování jakéhokoliv druhu otázek. Aplikace implementuje algoritmus Supermemo, který je založen na metodě postupného zvyšování intervalu dotazování na dané otázky. Při každé odpovědi program spočítá, kdy by si uživatel měl danou otázku zopakovat tak, aby odpověď byla správná a zároveň se co nejvíce zvyšoval interval mezi aktuální a předchozí odpovědí. Aplikace se adaptuje na schopnosti uživatele, v přiměřeném měřítku buď zvyšuje nebo snižuje interval dalšího připomenutí.

        % možné ještě rozvést aplikaci EasyWords

        % žádná z aplikací neumožňuje personalizované učení 
        % ve škole (z učebnice) se učí jiná slovíčka než v aplikacích
        % po otestování slova nedochází k jeho znovu připomenutí

        Z výše uvedených aplikací až na Supermemo žádná neumožňuje personalizovaný výběr učiva. Tedy nelze vložit vlastní slovíčko nebo si určit sadu slov pro testování. Proto jsou tyto aplikace především cílené pro uživatele, kteří vnímají výuku jazyka jako samostatné vzdělávání sami sebe. Pro studenty, kteří absolvují lekce z cizího jazyka ve škole je tento typ vzdělávání nevyhovující, jelikož se musí učit dvě nezávislé skupiny slov. Sice dochází k rozvinutí slovní zásoby studenta i do jiných okruhů než je jeho učebnice a málokterý student má ještě energii, časovou dotaci a vlastní iniciativu na to, aby se připravoval na školní test ze slovíček a ještě rozvíjel samostatně svoji cizojazyčnou slovní zásobu.

    \subsection{Učení slovíček}
        % problematika malých dětí a učení slov
        % Biemiller and Boote (2006)
        % ročně se dá zvládnout maximálně 400 slov u studentů 2 - 5 třídy
        % docházelo k zvýšení učenlivosti, když studenti si mohli zapsat 
        % slovo vlastní definicí
        Standardní učení slovní zásoby cizího jazyka je založeno na častém opakování slov. Dle výzkumu Biemillera a Boote se lze ročně zvládnout až 400 slov u studentů 3.—6. tříd \cite{bib:beimiller}. Což v případě cizího jazyka poměrně velké číslo, ale problém je, do jaké míry je slovo správně ukotveno v dlouhodobé paměti. Porovnání, kolik je potřeba slov pro ovládnutí anglického jazyka, usnadní následující tabulka \ref{tab:english-vocab-usage}, která zahrnuje procentuální využití nejvíce používaných slov v každém z odvětví \cite{bib:learning-vocab}. 

        \begin{table}[ht!]
            \centering
            \begin{tabular}{|l|c|c|c|}
            \hline
            & \multicolumn{1}{l|}{Konverzace} & \multicolumn{1}{l|}{Noviny} & \multicolumn{1}{l|}{Akademický text} \\ \hline
            1. 1000 slov & 84,3\% & 75,6\% & 73,5\% \\ \hline
            2. 1000 slov & 6\% & 4,7\% & 4,6\% \\ \hline
            Akademické výrazy& 1,9\% & 3,9\% & 8,5\% \\ \hline
            Ostatní & 7,8\% & 15,7\% & 13,3\% \\ \hline
            \end{tabular}
            \caption{Používání slov v britském anglickém jazyce}
            \label{tab:english-vocab-usage}
        \end{table}

        \subsubsection{Zastaralý způsob učení}
            Dle vlastního průzkumu žáci základních škol nevyužívají k učení slovíček nikterak pokročilé technologie. Většinou si udržují vlastní slovníček, do kterého nově nabytá slova. Při učení zakrývají část s cizími slovy a na základě českého ekvivalentu se snaží vybavit překlad slova. Tato metoda neposkytuje prakticky žádnou zpětnou vazbu. Žáci většinou pouze do krátkodobé paměti uloží slova, později při testu rychle zodpoví a následně během pár hodin si na slovo už ani nevzpomenou. Nevýhod a námětů na zlepšení metody má tato metoda celou řadu, ale jedním z klíčových vad je, že žáci si nevytvoří dostatečně souvislostí, aby řádně ukotvilo slovo v paměti.

        \subsubsection{Zvuková interpretace}
            Zejména v učení cizího jazyka je zvuková podoba a interpretace slov velice důležitá. Jelikož každý jazyk může hlásky různě zvukově interpretovat a pro nováčka v cizím jazyku nemusí textová výslovnost plně vyhovovat. Žák dále díky znění slova získá podvědomí o dialektu daného jazyka a zároveň dochází k lepšímu zapamatování slova. Právě díky zvukům dochází k propojení při učení i pravé mozkové hemisféry. A napomáhá tedy k vytvoření pevnější ukotvení v paměti. Dle studie bulharského vědce George Lozanova, který se zabýval studiem mozku a učebních metod, byl zjištěn obrovský přínos zvukových vjemů \cite{bib:suggestology}.

        \subsubsection{Problematika obtížnosti}
            % každý z žáků má indiviální úroveň znalostí cizího jazyka
            % a každý z žádů se jiným tempem učí cizojazyčná slovíčka
            Velkým problémem v učení slov je přizpůsobit obtížnost každému ze žáků individuálně. Jelikož ne všichni mají stejnou úroveň znalostí cizího jazyka a každý potřebuje jiné tempo pro zapamatování sady slov. Jsou žáci s výbornou pamětí, kterým stačí si slova pouze jednou projít a dokáží je používat, ale jsou žáci, kde nestačí je pětkrát zopakovat. Dalším problémem obtížnosti je z pohledu jednotlivých slov. Každé slovo má odlišnou obtížnost, které lze soudit například podle míry používanosti v jazyce, délky slova, zdali obsahuje přehlasování, dvojitá písmena nebo podobnost s referenčním slovem.

        \subsubsection{Učení slov v kontextu}
            % jak je důlžité se slova učit v kontextu - použití ve větě z učebnice
            % drive/vocab-techniques.pdf
            Pro učení slov cizího jazyka je rovněž důležité správné zasazení významu slova do kontextu. Dle průzkumu Biemillera a Bootea z roku 2006 bylo zjištěno, že u žáků od 10—13 let docházelo k nárůstu zapamatovaných slov o 4\%, pokud byla slova předložena v kontextu vět. Důležitým poznatkem z toho průzkumu je, že žáci si nejen déle naučené slovo pamatovali, ale správně ho i interpretovali, když měli za úkol ho svými slovy vysvětlit \cite{bib:beimiller}. Ve stejném průzkumu rovněž docházelo ještě k vyššímu zlepšení v učení, kdy si žáci zapisovali vlastními slovy definici a použití slova. 
        
    \subsection{Testování slovíček}
        % drive/accesing-vocabulary-in-the-language-classroom.pdf
        % důležitost aktivní slovní zásoby pro výuku cizího jazyka

        \subsubsection{Aktivní a pasivní slovní zásoba}
            % co je aktivní a pasivní
            Slovní zásobu, kterou využíváme k tvorbě vět ať už v cizím nebo mateřském jazyce rozdělujeme na dvě skupiny - aktivní a pasivní. Pasivní zásoba je sada slov, která jsou pevně a spolehlivě uložena v naší paměti. Průměrný žák zná cca 50 000 výrazů. Její velikost je ovlivněna věkem, vzděláním a četbou. Slova ze této sady využíváme zejména při písemné formě, ať už se jedná o čtení nebo psaní. Aktivní zásoba je sada slov, ve které lze najít žádané slovo během desítek milisekund. U většiny lidí dosahuje velikosti 4 000 až 8 000 výrazů \cite{bib:lexikologie}. Slovo se díky používání dostává z pasivní do aktivní slovní zásoby. 

        \subsubsection{Metody testování}
            \label{test-methods}
            % pasivní vs aktivní
            % aktivní - vzpomenutí, rozpoznání

            Metod testování slovní zásoby je mnoho. Základním rozdělením je na pasivní a aktivní. V případě pasivního se jedná například o výběr z nabídnutých možností. Jde o případ, kdy student nemusí znát přesnou odpověď a dokáže otázku vyhodnotit správně vylučovací metodou. Při aktivním testování žáci musí odpověď znát, aby otázka byla vyhodnocena správně. Pasivní metoda má pozitivní vliv například na motivaci žáka, který u ní tolik netápe a za pomoci zdravého rozumu může test vyhodnotit správně. Problém ale vzniká při používání získaných a procvičených informací. V případě cizího jazyka lze pasivní slovní zásobu využít pro čtení a náslech, ale pro ovládnutí cizího jazyka je nedostatečná a tudíž nevhodná. 

            Dalším rozdělením pasivního a aktivního testování slovíček je rozpoznávání (\textit{recognition}) a vzpomenutí (\textit{recall}). Rozpoznávání spočívá v předložení cizího slova žákovi a pro správné zodpovězení musí najít český ekvivalent. Vzpomenutí je způsob testování opačným způsobem než rozpoznání. Tedy uživateli je předložen výraz v jeho mateřském jazyce a on musí nalézt správný výraz v cizím. Rozpoznání je z principu jednodušší pro uživatele než vzpomenutí. Lze ho tedy využít rovněž pro zvýšení motivace při procvičování slov. Ale pro ověření, zda je slovo ovládnuté či nikoliv je metoda rozpoznání také nedostatečná.

        \subsection{Typy testů}
            % multiplechoice, matching, completion, translation
            % vyžití her - problém je, že tento způsob většinou není ani efektivní, jelikož si procvičí například v případě doplňování písmen pouze pár slov za poměrně dlouhý čas

            V analýze existujících řešení bylo procvičování a testování slov interpretováno různými způsoby. Obecně by se typ testů dal rozdělit na tři kategorie - textové testy, multimediální a herní testy.

            \subsubsection{Textové testy}
                Textové testy se vyskytovaly například jako výběr z více možností nebo spojováním spolu souvisejících významů, jak už ale bylo uvedeno v kapitole \ref{test-methods}, jedná se o rozvíjení pasivní slovní zásoby. Zajímavějším typem testů je doplňování slov do vět a klasický překlad slova. Tyto typy rozvíjejí žádanou aktivní slovní zásobu. Standardně textové testy byly doplněny zvukovou interpretací cizího slova.

            \subsubsection{Multimediální a herní testy}
                Zajímavým využitím multimédií by mohlo být zahrnutí otestování výslovnosti. Tedy možnost nahrání odpovědi a následně by došlo k rozpoznání. Ale analýza a rozpoznání řeči není nikterak triviální záležitost. Dalším zajímavým řešením byly herní testy. Díky nimž docházelo k zvýšením motivace uživatelů. Problém je, že tento typ procvičování většinou není zas tolik efektivní, jelikož si uživatel procvičí například v případě doplňování písmen křížovky nebo známé hry \textit{Hangman} pouze pár slov za poměrně dlouhý čas.  

    \subsection{Specifikace požadavků}
        % číselný seznam toho, co má aplikace dělat
        Na základě analýzy byla provedena specifikace požadavků, které budou implementovány v aplikaci pro testování slovíček z cizího jazyka.

        \begin{enumerate}
            \item správa učebnic
                \begin{itemize}
                    \item tvořit, editovat a mazat vlastní učebnice
                    \item umožnit publikovat učebnici pro veřejnost
                \end{itemize} 
            \item správa slovíček v učebnici
                \begin{itemize}
                    \item tvořit, editovat a mazat slovíčka
                    \item hromadně slovíčka do učebnice importovat
                    \item částečně automatizovat zvukovou a obrazovou interpretaci slovíčka
                \end{itemize} 
            \item správa modulů a tématických okruhů v učebnici
                \begin{itemize}
                    \item tvořit, editovat a mazat moduly a tématické okruhy učebnice
                    \item přiřazovat slovíčka do daných modulů a okruhů
                \end{itemize}
            \item správa uživatelských skupin (tříd)
                \begin{itemize}
                    \item tvořit, editovat a mazat uživatelské skupiny
                    \item umožnit uživatelům se přihlásit do skupiny
                \end{itemize}
            \item správa testovacích sad
                \begin{itemize}
                    \item tvořit, editovat a mazat testovací sady
                    \item umožnit vybírat slova pro testovací sadu z vlastních i veřejných učebnic
                    \item přiřazovat testovací sady ke skupinám uživatelů
                \end{itemize}
            \item procvičování slovíček
                \begin{itemize}
                    \item vybrat testovací sadu slovíček
                    \item generovat slova na základě úrovně uživatele
                    \item zahrnout obrazovou a zvukovou interpretaci do procvičování
                    \item možnost uložit stav testování a umožnit pozdější navázání
                \end{itemize}
            \item připomínat a procvičovat ovládnutou slovní zásobu 
            \item motivace
                \begin{itemize}
                    \item motivovat v rámci uživatelské skupiny
                    \item motivovat vlastní iniciativu k procvičování
                \end{itemize}
        \end{enumerate}


\newpage
\section{Návrh aplikace}

    \subsection{Uživatelské role a skupiny}
        % usecase diagram
        % jednotlivé use-case by měly být názvy jednotlivých podkapitol v návrhu aplikace
        % rozvést rozdíl mezi žákem a učitelem - rovnocenné partnerství, každý může být učitel a student
    
    \subsection{Učebnice}
        % sdílení učebnic (otevřený systém)
        % přijít do aplikace a jen tak si protestovat slova z učebnice
        % nemusí být člověk součástí žádné skupiny

        \subsection{Hromadný import slovíček}

    \subsection{Interpretace slovíček}
        % výhody více forem - lepší zapamatovatelnost
        % vytvoření hlubších asociací

        \subsubsection{Textová forma}
            % importováním sady slov (bez automatizace překladů)
            % editace definic a použití ve větách

        \subsubsection{Zvuková forma}
            % z Google API importovat zvukové nahrávky
            % shrnout omezení, problematiku

        \subsubsection{Obrazová forma}
            % vyhledání z Google Images API na základě cizojazyčného slova
            % autor učebnice vybírá dané slovo z importovaných
            % omezení dotazů - neprovádí se plně automatiky

    \subsection{Generování slov}
        % shrnout leitnerův algoritmus

        \subsubsection{Rozložené opakování} % Spaced repetition
            % obecně o metodě 

        \subsubsection{Leitnerův algoritmus}

        \subsubsection{Adaptivní a globální obtížnost}

    \subsection{Testování}

        \subsubsection{Metody testování}
            % aktivní rozpoznání a aktivní vzpomenutí
            % algoritmus na záměnu vzpomenutí a rozpoznání (možná obrázek)
    
        \subsubsection{Nápovědy}
            % definice se zobrazuje ihned, kontext slova ve větě z učebnice, první písmeno
            % možnost vyplnění definice slova - (automaticky předvyplnit - volně dostupné definice)

        \subsubsection{Kontrola podvádění}
            % problém, není možné kontrolovat, zda si uživatelé nevyhledávají
            % slova ve slovnících a potom nedoplňují do aplikace
            % měření času odpovědi - v závisloti na délce odpovědi

            % učitel nevidí statistiky dětí, pouze celé třídy
            % aplikace nemá sloužit na zkoušení dětí učitelem
            % testing to learn, not testing to assess        

    \subsection{Vyhodnocování odpovědí}
        % určení vzdálenosti slov - více úrovní odpovědí
        % shrnout levenstein algoritmus

    \subsection{Připomínání slov}
        % shrnout supermemo aplogritmus

    \subsection{Motivace}
        % počty zvládnutých slov studenta
        % třídních zvládnutých slov
        % cinknutí při zvládnutém slově

    \subsection{Model databáze}


    \subsection{Architektura aplikace}
        % architektura celé aplikace
        % 2 části, ve skratce popsat, která je za co zodpovědná


\newpage
\section{Klientská aplikace}

    \subsection{Návrh aplikace}

        \subsubsection{Single-page}
            % definice
            % výhody

        \subsubsection{Model aplikace}
            % url diagram
            % mock api

        \subsubsection{Editovatelné seznamy}
            % editace dat v podobě autosave editovatelných seznamů

        \subsubsection{Design}
            % wireframes
            % responzivnost      

        \subsubsection{Implementace algoritmů}
            % Levenstein a Leitner jsou implementovány na klientské straně

    \subsection{Vývojové prostředí}
        \subsubsection{Webpack}
            % css injections, production vlastní css soubor
            % - rychlejší načítání (solo js solo css)

        \subsubsection{Babel}
        \subsubsection{JSX}


    \subsection{Knihovna React}
        
        \subsubsection{Abstraktní DOM}
        \subsubsection{Mobx}
            % knihovna React se stává frameworkem
        \subsubsection{Imutabilita}
        \subsubsection{Směrování}
        \subsubsection{Komponenty}


    \subsection{Implementace technologií}
        % layout komponenta
        % využití Mobx - MVC (stores, components)

        \subsubsection{Adresářová struktura}
        

    \subsection{Testování}
        % Karma, NPM spouštění
        % využití MOBX data injections v daném stavu aplikace


\newpage
\section{Serverová aplikace}
    
    \subsection{Technologie}
        \subsubsection{Webová aplikace}
            % výhody webových aplikací

        \subsubsection{Architektura}
            % architektura klient + server
            % možnost více klientů (mobilní aplikace apod.)

        \subsubsection{HTTP a REST API}

        \subsubsection{WSGI}

        \subsubsection{Implementace algoritmů}
            % Supermemo implementace na serveru
            % CRON připomínací emaily pouze v případě, že stihneš dodělat do APP

    \subsection{Django a REST framework}
        
        \subsubsection{Autorizace}

        \subsubsection{Autentizace}

        \subsubsection{Optimalizace API}

    \subsection{Zabezpečení}

        \subsubsection{HTTP/2}

        \subsubsection{OAuth2}

        \subsubsection{JWT}

        \subsubsection{CORS}

    \subsection{PostgreSQL}

    \subsection{Testování}
        % APITestCase Django REST


\newpage
\section{Závěr}

\newpage
\begin{thebibliography}{99}
    
    \addcontentsline{toc}{section}{\refname}

    \bibitem{bib:terasoft}
        Terasoft, a.s. \textit{Terasoft - Výukové programy} [online] 2002-10-07. [cit. 2016-12-15]. Dostupné z: \url{http://www.terasoft.cz/czpages/cd_aj15.htm}
    
    \bibitem{bib:langsoft}
        LangSoft s.r.o. \textit{Language Teacher} [online]. [cit. 2016-12-07]. Dostupné z: \url{http://www.langsoft.cz/teacher.htm}

    \bibitem{bib:motivace}
        KREJČOVÁ, Lenka. \textit{Psychologické aspekty vzdělávání dospívajících}. 1. vyd. Praha: Grada Publishing, 2011 [cit 2016-12-18]. ISBN 978-80-247-3474-3.

    \bibitem{bib:beimiller}
        BIEMILLER, Andrew, BOOTE, Catherine. \textit{An effective method for building meaning vocabulary in primary grades}. Vol 98(1), Journal of Educational Psychology, 2006 [cit 2016-12-18].

    \bibitem{bib:suggestology}
        LOZANOV, Georgi. \textit{Suggestology and Outlines of Suggestopedy}. 1. vyd. Gordon and Breach, 1978 [cit 2016-12-18]. ISBN 0-203-39282-5.

    \bibitem{bib:learning-vocab}
        I. S. P. Nation \textit{Learning Vocabulary in Another Language}. Cambridge University Press, 2001 [cit 2016-12-18]. ISBN 0-521-800927.

    \bibitem{bib:lexikologie}
        SVOBODOVÁ Jana, SVOBODOVÁ Diana, KULDANOVÁ Pavlína, GEJGUŠOVÁ Ivana, ROSOVÁ Milena, NOVÁK Radomil. \textit{Lexikologie} [online] 2003. [cit. 2016-12-19]. Dostupné z: \url{http://www.osu.cz/fpd/kcd/dokumenty/cestinapositi/lexikologie.htm}
          

\end{thebibliography}
\end{document}

% !TeX spellcheck = cs_CZ
\documentclass[a4paper,11pt,titlepage,fleqn]{article}

\usepackage[utf8]{inputenc}
\usepackage[top=3.5cm, bottom=3cm, left=3.5cm, right=2.5cm]{geometry}
\usepackage[czech]{babel}
\usepackage[IL2]{fontenc}  
\usepackage{graphicx}
\usepackage[nonumberlist,acronym]{glossaries}
\usepackage{cite}
\usepackage{fancyhdr}
\usepackage{afterpage}
\usepackage[hidelinks,unicode,hyperfootnotes]{hyperref}
\usepackage{footnote}
\usepackage{parskip}
\usepackage{setspace} 
\usepackage{listings}
\usepackage{pdfpages}
\usepackage{dirtree}
%\usepackage[T1]{fontenc}
%\usepackage[bottom]{footmisc}
%\usepackage{fancyvrb}

\lstset{
    language=PHP,
    basicstyle=\ttfamily\small,
    breaklines=true,
    prebreak=\raisebox{0ex}[0ex][0ex]{\ensuremath{\hookleftarrow}},
    frame=lines,
    showtabs=false,
    showspaces=false,
    showstringspaces=false,
    keywordstyle=\color{red}\bfseries,
    stringstyle=\color{green!50!black},
    commentstyle=\color{gray}\itshape,
    numbers=left,
    captionpos=t,
    escapeinside={\%*}{*)}
}

\renewcommand{\lstlistingname}{Ukázka kódu}
\renewcommand*{\lstlistlistingname}{Seznam zdrojových kódů}

\makeglossaries
\renewcommand*{\glsgroupskip}{}

%\fancyhead[RE,RO]{\textsc{\nouppercase{\leftmark}}}
\rhead{\textsc{\nouppercase{\leftmark}}}
\lhead{}
\pagestyle{fancy}

\addto\captionsczech{\def\refname{Použitá literatura}}
\linespread{1.3}
\setlength{\headheight}{15pt}

\newglossaryentry{tulg}{
    name={Technická univerzita v~Liberci},
    description={Technická univerzita v~Liberci}
}
\newglossaryentry{dp}{
    type=\acronymtype,
    name={DP},
    description={Diplomová práce},
    first={DP}
}

\begin{document}
\includepdf[pages={1,2,3}]{dp-titlepage.pdf}
\includepdf{prohlaseni.pdf}
\setcounter{page}{3}

\newpage
\thispagestyle{plain}
\section*{Abstrakt}


\section*{Klíčová slova}


\thispagestyle{empty}
\newpage

\section*{Abstract}

\section*{Keywords}


\thispagestyle{empty}

\newpage
\tableofcontents
\newpage
\listoffigures
\listoftables
\lstlistoflistings
\newpage
\printglossary[type=\acronymtype,title=Seznam zkratek]
%\printglossary[style=altlist,title=Slovník]
\cleardoublepage


\section{Úvod}


\newpage
\section{Kapitola 1}

\newpage
\section{Kapitola 2}
    \label{kapitola}

    \subsection{Podkapitola 2.1} 
        \label{kapitola21}

        % \subsubsection{}
        % \begin{figure}[ht!]
        %   \centering
        %   \includegraphics[scale=0.65]{file.pdf}
        %   \caption{}
        %   \label{fig:name}
        % \end{figure}

\newpage
\section{Závěr}

\newpage
\begin{thebibliography}{99}
\addcontentsline{toc}{section}{\refname}

% \bibitem{5EKmCBUoyUxNVwii}
% CAKE DEVELOPMENT CORPORATION. \textit{CakePHP: The rapid application development framework for PHP} [online]. [cit. 2014-04-05]. Dostupné z: \url{http://cakephp.org/}

% \bibitem{Gilmore2007}
% GILMORE, Jason W. \textit{Velká kniha PHP a~MySQL 5: kompendium znalostí pro začátečníky i~profesionály}. Vyd.~1. [i.e. 2.~vyd.]. Brno: Zoner Press, 2007, 864~s. ISBN 80-868-1553-6. 

\end{thebibliography}

% \newpage
% \section*{Obsah přiloženého DVD}
% \addcontentsline{toc}{section}{Obsah přiloženého DVD}

% \begin{figure}[ht!]
%     \dirtree{
%         .1 doc.
%         .2 pdf.
%         .3 dp\_sss.pdf.
%         .3 zadani\_madera.pdf.
%         .2 src.
%         .3 dp\_madera.tex.
%         .3 dp\_titlepage.pdf.
%         .3 \dots.
%         .1 src.
%         .2 app.
%         .3 html.
%         .4 index.php.
%         .3 login.
%         .4 app.
%         .4 \dots.
%         .3 README.
%     }
% \end{figure}

% \begin{itemize}
%     \item {\texttt{doc} -- podadresář \texttt{pdf} obsahuje zadání a kompletní text DP ve formátu PDF, podadresář \verb|src| pak {\LaTeX}ové \uv{zdrojové kódy} a~obrázky použité v~textu DP.}
%     \item {\texttt{src} -- podadresář \texttt{app} obsahuje zdrojové kódy webové aplikce}
% \end{itemize}

\end{document}

% !TeX spellcheck = cs_CZ
\documentclass[a4paper,11pt,titlepage,fleqn]{article}

\usepackage[utf8]{inputenc}
\usepackage[top=3cm, bottom=2.5cm, left=3.5cm, right=2.5cm]{geometry}
\usepackage[czech]{babel}
\usepackage[IL2]{fontenc}  
\usepackage{graphicx}
\usepackage[nonumberlist,acronym]{glossaries}
\usepackage{cite}
\usepackage{fancyhdr}
\usepackage{afterpage}
\usepackage[hidelinks,unicode,hyperfootnotes]{hyperref}
\usepackage{footnote}
\usepackage{parskip}
\usepackage{setspace} 
\usepackage{listings}
\usepackage{pdfpages}
\usepackage{dirtree}
%\usepackage[T1]{fontenc}
%\usepackage[bottom]{footmisc}
%\usepackage{fancyvrb}

\lstset{
    language=PHP,
    basicstyle=\ttfamily\small,
    breaklines=true,
    prebreak=\raisebox{0ex}[0ex][0ex]{\ensuremath{\hookleftarrow}},
    frame=lines,
    showtabs=false,
    showspaces=false,
    showstringspaces=false,
    keywordstyle=\color{red}\bfseries,
    stringstyle=\color{green!50!black},
    commentstyle=\color{gray}\itshape,
    numbers=left,
    captionpos=t,
    escapeinside={\%*}{*)}
}

\renewcommand{\lstlistingname}{Ukázka kódu}
\renewcommand*{\lstlistlistingname}{Seznam zdrojových kódů}

\makeglossaries
\renewcommand*{\glsgroupskip}{}

%\fancyhead[RE,RO]{\textsc{\nouppercase{\leftmark}}}
\rhead{\textsc{\nouppercase{\leftmark}}}
\lhead{}
\pagestyle{fancy}

\addto\captionsczech{\def\refname{Použitá literatura}}
\linespread{1.3}
\setlength{\headheight}{15pt}

\newglossaryentry{tulg}{
    name={Technická univerzita v~Liberci},
    description={Technická univerzita v~Liberci}
}
\newglossaryentry{dp}{
    type=\acronymtype,
    name={DP},
    description={Diplomová práce},
    first={DP}
}

\begin{document}
\includepdf[pages={1,2,3}]{dp-titlepage.pdf}
\includepdf{prohlaseni.pdf}
\setcounter{page}{3}

\newpage
\thispagestyle{plain}
\section*{Abstrakt}
% 250 až 500 slov

\section*{Klíčová slova}
% 5 klíčových slov

\thispagestyle{empty}
\newpage

\section*{Abstract}
% 250 až 500 slov

\section*{Keywords}
% 5 klíčových slov

\thispagestyle{empty}

\newpage
\tableofcontents

\newpage
\listoffigures
\listoftables
\lstlistoflistings
\newpage
\printglossary[type=\acronymtype,title=Seznam zkratek]
%\printglossary[style=altlist,title=Slovník]
\cleardoublepage


\section{Úvod}
    % proč se zaobývat tímto tématem
    % neopakovat abstrakt, lehce nastínit zadání, jaká je motivace
    % popsat, jak je práce strukturovaná - rozcestník


\newpage
\section{Analýza}
    % problematika malých dětí a učení slov

    \subsection{Hlavní cíle}
        % jak si aplikaci představuji
        % motivace dětí k učení slov (vrámci třídy)
        % zlepšení přípravy na hodiny cizího jazyka
        % shrnout v pár odstavcích požadavky na aplikaci


    \subsection{Existující řešení}
        % Langsoft Teacher langsoft.cz
        % Duolingo
        % Languagecourse.net - Vocabulary Trainer
        % AnkiDroid (usermanual)
        % Supermemo aplikace

        % žádná z aplikací neumožňuje personalizované učení 
        % ve škole (z učebnice) se učí jiná slovíčka než v aplikacích
        % po otestování slova nedochází k jeho znovu připomenutí

    \subsection{Učení slovíček}
        % Biemiller and Boote (2006)
        % ročně se dá zvládnout maximálně 400 slov u studentů 2 - 5 třídy
        % docházelo k zvýšení učenlivosti, když studenti si mohli zapsat 
        % slovo vlastní definicí

        \subsubsection{Učení slov v kontextu}
        % jak je důlžité se slova učit v kontextu
        % drive/vocab-techniques.pdf
    
    \subsection{Testování slovíček}
        % drive/accesing-vocabulary-in-the-language-classroom.pdf
        % pasivní vs aktivní
        % aktivní - vzpomenutí, rozpoznání
        % důležitost aktivní slovní zásoby pro výuku cizího jazyka

        % multiplechoice, matching, completion, translation
    

\newpage
\section{Návrh aplikace}
    
    \subsection{Učebnice}
    % sdílení učebnic (otevřený systém)
    % přijít do aplikace a jen tak si protestovat slova z učebnice
    % nemusí být člověk součástí žádné skupiny

    \subsection{Uživatelské role a skupiny}
    % možná use-case diagram

    \subsection{Testování slov}

        \subsubsection{Nápovědy}
            % definice se zobrazuje ihned, kontext slova ve větě z učebnice, první písmeno
            % možnost vyplnění definice slova - (automaticky předvyplnit - volně dostupné definice)

        \subsubsection{Kontrola podvádění}
            % problém, není možné kontrolovat, zda si uživatelé nevyhledávají
            % slova ve slovnících a potom nedoplňují do aplikace
            % měření času odpovědi - v závisloti na délce odpovědi

    \subsection{Interpretace slovíček}
        % výhody více forem - lepší zapamatovatelnost
        % vytvoření hlubších asociací

        \subsubsection{Textová forma}
            % importováním sady slov (bez automatizace překladů)

        \subsubsection{Zvuková forma}
            % z Google API importovat zvukové nahrávky
            % shrnout omezení, problematiku

        \subsubsection{Obrazová forma}
            % vyhledání z Google Images API na základě cizojazyčného slova
            % autor učebnice vybírá dané slovo z importovaných
            % omezení dotazů - neprovádí se plně automatiky

    \subsection{Generování slov}
        % shrnout leitnerův algoritmus

        \subsubsection{Obtížnost}
        \subsubsection{Adaptivní obtížnost}
        

    \subsection{Vyhodnocování odpovědí}
        % určení vzdálenosti slov - více úrovní odpovědí
        % shrnout levenstein algoritmus

    \subsection{Připomínání slov}
        % shrnout supermemo aplogritmus


\newpage
\section{Klientská aplikace}

    \subsection{Design}

    \subsection{Vývojové prostředí}
        \subsubsection{Webpack}
            % css injections, production vlastní css soubor
            % - rychlejší načítání (solo js solo css)

        \subsubsection{Babel}
        \subsubsection{JSX}


    \subsection{React}
        
        \subsubsection{Abstraktní DOM}
        \subsubsection{Mobx}
        \subsubsection{Imutabilita}

    \subsection{Testování}


\newpage
\section{Serverová aplikace}
    
    \subsection{Technologie}
        \subsubsection{Webová aplikace}
            % výhody webových aplikací

        \subsubsection{Architektura}
            % architektura klient + server
            % možnost více klientů (mobilní aplikace apod.)

    \subsection{Django}
        \subsubsection{Django REST Framework}
        \subsubsection{Optimalizace API}

    \subsection{Zabezpečení}
        \subsubsection{HTTP/2}
        \subsubsection{OAuth2}
        \subsubsection{JWT}

    \subsection{Testování}

    \subsection{PostgreSQL}

\newpage
\section{Závěr}

\newpage
\begin{thebibliography}{99}
\addcontentsline{toc}{section}{\refname}
\end{thebibliography}
\end{document}
